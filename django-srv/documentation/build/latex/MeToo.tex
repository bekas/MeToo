% Generated by Sphinx.
\def\sphinxdocclass{report}
\documentclass[letterpaper,10pt,english]{sphinxmanual}
\usepackage[utf8]{inputenc}
\DeclareUnicodeCharacter{00A0}{\nobreakspace}
\usepackage[T1]{fontenc}
\usepackage{babel}
\usepackage{times}
\usepackage[Bjarne]{fncychap}
\usepackage{longtable}
\usepackage{sphinx}
\usepackage{multirow}


\title{MeToo Documentation}
\date{December 02, 2012}
\release{091}
\author{MeTooCompany}
\newcommand{\sphinxlogo}{}
\renewcommand{\releasename}{Release}
\makeindex

\makeatletter
\def\PYG@reset{\let\PYG@it=\relax \let\PYG@bf=\relax%
    \let\PYG@ul=\relax \let\PYG@tc=\relax%
    \let\PYG@bc=\relax \let\PYG@ff=\relax}
\def\PYG@tok#1{\csname PYG@tok@#1\endcsname}
\def\PYG@toks#1+{\ifx\relax#1\empty\else%
    \PYG@tok{#1}\expandafter\PYG@toks\fi}
\def\PYG@do#1{\PYG@bc{\PYG@tc{\PYG@ul{%
    \PYG@it{\PYG@bf{\PYG@ff{#1}}}}}}}
\def\PYG#1#2{\PYG@reset\PYG@toks#1+\relax+\PYG@do{#2}}

\def\PYG@tok@gd{\def\PYG@tc##1{\textcolor[rgb]{0.63,0.00,0.00}{##1}}}
\def\PYG@tok@gu{\let\PYG@bf=\textbf\def\PYG@tc##1{\textcolor[rgb]{0.50,0.00,0.50}{##1}}}
\def\PYG@tok@gt{\def\PYG@tc##1{\textcolor[rgb]{0.00,0.25,0.82}{##1}}}
\def\PYG@tok@gs{\let\PYG@bf=\textbf}
\def\PYG@tok@gr{\def\PYG@tc##1{\textcolor[rgb]{1.00,0.00,0.00}{##1}}}
\def\PYG@tok@cm{\let\PYG@it=\textit\def\PYG@tc##1{\textcolor[rgb]{0.25,0.50,0.56}{##1}}}
\def\PYG@tok@vg{\def\PYG@tc##1{\textcolor[rgb]{0.73,0.38,0.84}{##1}}}
\def\PYG@tok@m{\def\PYG@tc##1{\textcolor[rgb]{0.13,0.50,0.31}{##1}}}
\def\PYG@tok@mh{\def\PYG@tc##1{\textcolor[rgb]{0.13,0.50,0.31}{##1}}}
\def\PYG@tok@cs{\def\PYG@tc##1{\textcolor[rgb]{0.25,0.50,0.56}{##1}}\def\PYG@bc##1{\colorbox[rgb]{1.00,0.94,0.94}{##1}}}
\def\PYG@tok@ge{\let\PYG@it=\textit}
\def\PYG@tok@vc{\def\PYG@tc##1{\textcolor[rgb]{0.73,0.38,0.84}{##1}}}
\def\PYG@tok@il{\def\PYG@tc##1{\textcolor[rgb]{0.13,0.50,0.31}{##1}}}
\def\PYG@tok@go{\def\PYG@tc##1{\textcolor[rgb]{0.19,0.19,0.19}{##1}}}
\def\PYG@tok@cp{\def\PYG@tc##1{\textcolor[rgb]{0.00,0.44,0.13}{##1}}}
\def\PYG@tok@gi{\def\PYG@tc##1{\textcolor[rgb]{0.00,0.63,0.00}{##1}}}
\def\PYG@tok@gh{\let\PYG@bf=\textbf\def\PYG@tc##1{\textcolor[rgb]{0.00,0.00,0.50}{##1}}}
\def\PYG@tok@ni{\let\PYG@bf=\textbf\def\PYG@tc##1{\textcolor[rgb]{0.84,0.33,0.22}{##1}}}
\def\PYG@tok@nl{\let\PYG@bf=\textbf\def\PYG@tc##1{\textcolor[rgb]{0.00,0.13,0.44}{##1}}}
\def\PYG@tok@nn{\let\PYG@bf=\textbf\def\PYG@tc##1{\textcolor[rgb]{0.05,0.52,0.71}{##1}}}
\def\PYG@tok@no{\def\PYG@tc##1{\textcolor[rgb]{0.38,0.68,0.84}{##1}}}
\def\PYG@tok@na{\def\PYG@tc##1{\textcolor[rgb]{0.25,0.44,0.63}{##1}}}
\def\PYG@tok@nb{\def\PYG@tc##1{\textcolor[rgb]{0.00,0.44,0.13}{##1}}}
\def\PYG@tok@nc{\let\PYG@bf=\textbf\def\PYG@tc##1{\textcolor[rgb]{0.05,0.52,0.71}{##1}}}
\def\PYG@tok@nd{\let\PYG@bf=\textbf\def\PYG@tc##1{\textcolor[rgb]{0.33,0.33,0.33}{##1}}}
\def\PYG@tok@ne{\def\PYG@tc##1{\textcolor[rgb]{0.00,0.44,0.13}{##1}}}
\def\PYG@tok@nf{\def\PYG@tc##1{\textcolor[rgb]{0.02,0.16,0.49}{##1}}}
\def\PYG@tok@si{\let\PYG@it=\textit\def\PYG@tc##1{\textcolor[rgb]{0.44,0.63,0.82}{##1}}}
\def\PYG@tok@s2{\def\PYG@tc##1{\textcolor[rgb]{0.25,0.44,0.63}{##1}}}
\def\PYG@tok@vi{\def\PYG@tc##1{\textcolor[rgb]{0.73,0.38,0.84}{##1}}}
\def\PYG@tok@nt{\let\PYG@bf=\textbf\def\PYG@tc##1{\textcolor[rgb]{0.02,0.16,0.45}{##1}}}
\def\PYG@tok@nv{\def\PYG@tc##1{\textcolor[rgb]{0.73,0.38,0.84}{##1}}}
\def\PYG@tok@s1{\def\PYG@tc##1{\textcolor[rgb]{0.25,0.44,0.63}{##1}}}
\def\PYG@tok@gp{\let\PYG@bf=\textbf\def\PYG@tc##1{\textcolor[rgb]{0.78,0.36,0.04}{##1}}}
\def\PYG@tok@sh{\def\PYG@tc##1{\textcolor[rgb]{0.25,0.44,0.63}{##1}}}
\def\PYG@tok@ow{\let\PYG@bf=\textbf\def\PYG@tc##1{\textcolor[rgb]{0.00,0.44,0.13}{##1}}}
\def\PYG@tok@sx{\def\PYG@tc##1{\textcolor[rgb]{0.78,0.36,0.04}{##1}}}
\def\PYG@tok@bp{\def\PYG@tc##1{\textcolor[rgb]{0.00,0.44,0.13}{##1}}}
\def\PYG@tok@c1{\let\PYG@it=\textit\def\PYG@tc##1{\textcolor[rgb]{0.25,0.50,0.56}{##1}}}
\def\PYG@tok@kc{\let\PYG@bf=\textbf\def\PYG@tc##1{\textcolor[rgb]{0.00,0.44,0.13}{##1}}}
\def\PYG@tok@c{\let\PYG@it=\textit\def\PYG@tc##1{\textcolor[rgb]{0.25,0.50,0.56}{##1}}}
\def\PYG@tok@mf{\def\PYG@tc##1{\textcolor[rgb]{0.13,0.50,0.31}{##1}}}
\def\PYG@tok@err{\def\PYG@bc##1{\fcolorbox[rgb]{1.00,0.00,0.00}{1,1,1}{##1}}}
\def\PYG@tok@kd{\let\PYG@bf=\textbf\def\PYG@tc##1{\textcolor[rgb]{0.00,0.44,0.13}{##1}}}
\def\PYG@tok@ss{\def\PYG@tc##1{\textcolor[rgb]{0.32,0.47,0.09}{##1}}}
\def\PYG@tok@sr{\def\PYG@tc##1{\textcolor[rgb]{0.14,0.33,0.53}{##1}}}
\def\PYG@tok@mo{\def\PYG@tc##1{\textcolor[rgb]{0.13,0.50,0.31}{##1}}}
\def\PYG@tok@mi{\def\PYG@tc##1{\textcolor[rgb]{0.13,0.50,0.31}{##1}}}
\def\PYG@tok@kn{\let\PYG@bf=\textbf\def\PYG@tc##1{\textcolor[rgb]{0.00,0.44,0.13}{##1}}}
\def\PYG@tok@o{\def\PYG@tc##1{\textcolor[rgb]{0.40,0.40,0.40}{##1}}}
\def\PYG@tok@kr{\let\PYG@bf=\textbf\def\PYG@tc##1{\textcolor[rgb]{0.00,0.44,0.13}{##1}}}
\def\PYG@tok@s{\def\PYG@tc##1{\textcolor[rgb]{0.25,0.44,0.63}{##1}}}
\def\PYG@tok@kp{\def\PYG@tc##1{\textcolor[rgb]{0.00,0.44,0.13}{##1}}}
\def\PYG@tok@w{\def\PYG@tc##1{\textcolor[rgb]{0.73,0.73,0.73}{##1}}}
\def\PYG@tok@kt{\def\PYG@tc##1{\textcolor[rgb]{0.56,0.13,0.00}{##1}}}
\def\PYG@tok@sc{\def\PYG@tc##1{\textcolor[rgb]{0.25,0.44,0.63}{##1}}}
\def\PYG@tok@sb{\def\PYG@tc##1{\textcolor[rgb]{0.25,0.44,0.63}{##1}}}
\def\PYG@tok@k{\let\PYG@bf=\textbf\def\PYG@tc##1{\textcolor[rgb]{0.00,0.44,0.13}{##1}}}
\def\PYG@tok@se{\let\PYG@bf=\textbf\def\PYG@tc##1{\textcolor[rgb]{0.25,0.44,0.63}{##1}}}
\def\PYG@tok@sd{\let\PYG@it=\textit\def\PYG@tc##1{\textcolor[rgb]{0.25,0.44,0.63}{##1}}}

\def\PYGZbs{\char`\\}
\def\PYGZus{\char`\_}
\def\PYGZob{\char`\{}
\def\PYGZcb{\char`\}}
\def\PYGZca{\char`\^}
\def\PYGZsh{\char`\#}
\def\PYGZpc{\char`\%}
\def\PYGZdl{\char`\$}
\def\PYGZti{\char`\~}
% for compatibility with earlier versions
\def\PYGZat{@}
\def\PYGZlb{[}
\def\PYGZrb{]}
\makeatother

\begin{document}

\maketitle
\tableofcontents
\phantomsection\label{index::doc}


Contents:


\chapter{CHAPTER 1 MeToo}
\label{intr::doc}\label{intr:welcome-to-metoo-s-documentation}\label{intr:chapter-1-metoo}
\includegraphics{metoo2.png}


\section{Введение}
\label{intr:id1}
Геолокационный сервис для организации собраний групп людей, исповедующих единую цель времяпрепровождения.

\begin{notice}{note}{Note:}
Авторы: Бекасов ИУ7-17, Бурмистрова ИУ7-17, Флис ИУ7-17
\end{notice}


\section{Мотивация}
\label{intr:id2}
Существует множество сервисов, облегчающих пользователю организацию личного досуга - все они являются социальными сетями. Одни социальные сети (как ВКонтакте, Facebook) позволяют создавать группы пользователей с общими интересами, а также организовывать группы-“встречи” для отдельных событий. Другие (как foursquare, Яндекс.Карты и тот же Facebook) позволяют пользователям обозначать на карте интересные места, кафе и прочие заведения, оставлять отзывы о них а также сообщать своим друзьям о том, где они находятся сейчас.

Однако на стыке уже реализованных идей есть незанятая ниша, которая может быть занята данным проектом. Пользователю было бы удобно видеть на карте не просто интересные места и заведения, но и события, интересные людям. Сейчас пользователи интернета при принятии решения о том, где провести свободное время, вынуждены посетить множество сайтов, посвящённых аггрегации мероприятий.

Выбор осложняется тем, что большинство сервисов не оснащено удобной картой: пользователь может испытывать неудобства при выборе маршрута, если он захочет посетить несколько мероприятий. Также возникают трудности с рекламой небольших мероприятий, информация о которых просто тонет в современных информационных потоках и имеет мало шансов дойти до целевой аудитории.

Эти проблемы можно решить с помощью разработки нового геоориентированного сервиса, учитывающего недостатки существующих систем.


\chapter{CHAPTER 2 Модель данных приложения}
\label{models:chapter-2}\label{models::doc}

\section{Модель}
\label{models:module-mainServer.models}\label{models:id1}\index{mainServer.models (module)}
Модуль описания модели данных сервера
\index{City (class in mainServer.models)}

\begin{fulllineitems}
\phantomsection\label{models:mainServer.models.City}\pysiglinewithargsret{\strong{class }\code{mainServer.models.}\bfcode{City}}{\emph{*args}, \emph{**kwargs}}{}
Класс описания сущности ``Город''

\end{fulllineitems}

\index{Country (class in mainServer.models)}

\begin{fulllineitems}
\phantomsection\label{models:mainServer.models.Country}\pysiglinewithargsret{\strong{class }\code{mainServer.models.}\bfcode{Country}}{\emph{*args}, \emph{**kwargs}}{}
Класс описания сущности ``Страна''

\end{fulllineitems}

\index{Event (class in mainServer.models)}

\begin{fulllineitems}
\phantomsection\label{models:mainServer.models.Event}\pysiglinewithargsret{\strong{class }\code{mainServer.models.}\bfcode{Event}}{\emph{*args}, \emph{**kwargs}}{}
Класс описания сущности ``Событие''

\end{fulllineitems}

\index{EventType (class in mainServer.models)}

\begin{fulllineitems}
\phantomsection\label{models:mainServer.models.EventType}\pysiglinewithargsret{\strong{class }\code{mainServer.models.}\bfcode{EventType}}{\emph{*args}, \emph{**kwargs}}{}
Класс описания сущности ``Тип события''

\end{fulllineitems}

\index{Friend (class in mainServer.models)}

\begin{fulllineitems}
\phantomsection\label{models:mainServer.models.Friend}\pysiglinewithargsret{\strong{class }\code{mainServer.models.}\bfcode{Friend}}{\emph{*args}, \emph{**kwargs}}{}
Класс описания дружественной связи ``Пользователь'' - ``Пользователь''

\end{fulllineitems}

\index{Interest (class in mainServer.models)}

\begin{fulllineitems}
\phantomsection\label{models:mainServer.models.Interest}\pysiglinewithargsret{\strong{class }\code{mainServer.models.}\bfcode{Interest}}{\emph{*args}, \emph{**kwargs}}{}
Класс описания сущности ``Интерес''

\end{fulllineitems}

\index{Metoo (class in mainServer.models)}

\begin{fulllineitems}
\phantomsection\label{models:mainServer.models.Metoo}\pysiglinewithargsret{\strong{class }\code{mainServer.models.}\bfcode{Metoo}}{\emph{*args}, \emph{**kwargs}}{}
Класс описания связи ``Пользователь'' - ``Событие''

\end{fulllineitems}

\index{MetooType (class in mainServer.models)}

\begin{fulllineitems}
\phantomsection\label{models:mainServer.models.MetooType}\pysiglinewithargsret{\strong{class }\code{mainServer.models.}\bfcode{MetooType}}{\emph{*args}, \emph{**kwargs}}{}
Класс описания сущности ``Тип похода на событие''

\end{fulllineitems}

\index{Photo (class in mainServer.models)}

\begin{fulllineitems}
\phantomsection\label{models:mainServer.models.Photo}\pysiglinewithargsret{\strong{class }\code{mainServer.models.}\bfcode{Photo}}{\emph{*args}, \emph{**kwargs}}{}
Класс описания сущности ``Фото''

\end{fulllineitems}

\index{Place (class in mainServer.models)}

\begin{fulllineitems}
\phantomsection\label{models:mainServer.models.Place}\pysiglinewithargsret{\strong{class }\code{mainServer.models.}\bfcode{Place}}{\emph{*args}, \emph{**kwargs}}{}
Класс описания сущности ``Место''

\end{fulllineitems}

\index{Session (class in mainServer.models)}

\begin{fulllineitems}
\phantomsection\label{models:mainServer.models.Session}\pysiglinewithargsret{\strong{class }\code{mainServer.models.}\bfcode{Session}}{\emph{*args}, \emph{**kwargs}}{}
Класс описания сущности ``Сессия''

\end{fulllineitems}

\index{SocialNetwork (class in mainServer.models)}

\begin{fulllineitems}
\phantomsection\label{models:mainServer.models.SocialNetwork}\pysiglinewithargsret{\strong{class }\code{mainServer.models.}\bfcode{SocialNetwork}}{\emph{*args}, \emph{**kwargs}}{}
Класс описания сущности ``Социальная сеть''

\end{fulllineitems}

\index{User (class in mainServer.models)}

\begin{fulllineitems}
\phantomsection\label{models:mainServer.models.User}\pysiglinewithargsret{\strong{class }\code{mainServer.models.}\bfcode{User}}{\emph{*args}, \emph{**kwargs}}{}
Класс описания сущности ``Пользователь''

\end{fulllineitems}

\index{UserInterest (class in mainServer.models)}

\begin{fulllineitems}
\phantomsection\label{models:mainServer.models.UserInterest}\pysiglinewithargsret{\strong{class }\code{mainServer.models.}\bfcode{UserInterest}}{\emph{*args}, \emph{**kwargs}}{}
Класс описания связи ``Пользователь'' - ``Интерес''

\end{fulllineitems}

\index{UserSocialNetwork (class in mainServer.models)}

\begin{fulllineitems}
\phantomsection\label{models:mainServer.models.UserSocialNetwork}\pysiglinewithargsret{\strong{class }\code{mainServer.models.}\bfcode{UserSocialNetwork}}{\emph{*args}, \emph{**kwargs}}{}
Класс описания связи ``Пользователь'' - ``Социальная сеть''

\end{fulllineitems}



\chapter{CHAPTER 3 Представление (View)}
\label{view:chapter-3-view}\label{view::doc}

\section{Представление}
\label{view:id1}\phantomsection\label{view:module-mainServer.views}\index{mainServer.views (module)}
Модуль работы с представлениями.
\index{aboutPage() (in module mainServer.views)}

\begin{fulllineitems}
\phantomsection\label{view:mainServer.views.aboutPage}\pysiglinewithargsret{\code{mainServer.views.}\bfcode{aboutPage}}{\emph{request}}{}
Метод, возвращающий страницу сайта ``About''

\end{fulllineitems}

\index{devPage() (in module mainServer.views)}

\begin{fulllineitems}
\phantomsection\label{view:mainServer.views.devPage}\pysiglinewithargsret{\code{mainServer.views.}\bfcode{devPage}}{\emph{request}}{}
Метод, возвращающий страницу сайта ``Developer page'' для отладки

\end{fulllineitems}

\index{downloadPage() (in module mainServer.views)}

\begin{fulllineitems}
\phantomsection\label{view:mainServer.views.downloadPage}\pysiglinewithargsret{\code{mainServer.views.}\bfcode{downloadPage}}{\emph{request}}{}
Метод, возвращающий страницу сайта ``Download''

\end{fulllineitems}

\index{getAgentVersion() (in module mainServer.views)}

\begin{fulllineitems}
\phantomsection\label{view:mainServer.views.getAgentVersion}\pysiglinewithargsret{\code{mainServer.views.}\bfcode{getAgentVersion}}{\emph{userAgent}}{}
Метод - заглушка, возвращающий версию агента

\end{fulllineitems}

\index{mainPage() (in module mainServer.views)}

\begin{fulllineitems}
\phantomsection\label{view:mainServer.views.mainPage}\pysiglinewithargsret{\code{mainServer.views.}\bfcode{mainPage}}{\emph{request}}{}
Метод, возвращающий основную страницу сайта

\end{fulllineitems}

\index{processAgent() (in module mainServer.views)}

\begin{fulllineitems}
\phantomsection\label{view:mainServer.views.processAgent}\pysiglinewithargsret{\code{mainServer.views.}\bfcode{processAgent}}{\emph{userAgent}}{}
Метод анализирующий агента

\end{fulllineitems}

\index{processRequest() (in module mainServer.views)}

\begin{fulllineitems}
\phantomsection\label{view:mainServer.views.processRequest}\pysiglinewithargsret{\code{mainServer.views.}\bfcode{processRequest}}{\emph{request}, \emph{typePacket}}{}
Метод обработки запроса

\end{fulllineitems}

\index{render() (in module mainServer.views)}

\begin{fulllineitems}
\phantomsection\label{view:mainServer.views.render}\pysiglinewithargsret{\code{mainServer.views.}\bfcode{render}}{\emph{context}, \emph{packet}, \emph{mode=False}}{}
Метод выполняющий рендер страницы

\end{fulllineitems}

\index{whatPage() (in module mainServer.views)}

\begin{fulllineitems}
\phantomsection\label{view:mainServer.views.whatPage}\pysiglinewithargsret{\code{mainServer.views.}\bfcode{whatPage}}{\emph{request}}{}
Метод, возвращающий страницу сайта ``What is it''

\end{fulllineitems}



\chapter{CHAPTER 4 Message manager}
\label{messageManager::doc}\label{messageManager:chapter-4-message-manager}

\section{Система управления пакетами и сообщениями}
\label{messageManager:id1}\phantomsection\label{messageManager:module-mainServer.messageManager}\index{mainServer.messageManager (module)}
Модуль, отвечающий за работу с событиями.
\index{MessageManager (class in mainServer.messageManager)}

\begin{fulllineitems}
\phantomsection\label{messageManager:mainServer.messageManager.MessageManager}\pysigline{\strong{class }\code{mainServer.messageManager.}\bfcode{MessageManager}}
Класс, обрабатывающий входные сообщения и формирующий ответы.

Основной метод - createContext.Создает словарь, содержащий необходимую

информацию
\index{aboutContext() (mainServer.messageManager.MessageManager static method)}

\begin{fulllineitems}
\phantomsection\label{messageManager:mainServer.messageManager.MessageManager.aboutContext}\pysiglinewithargsret{\strong{static }\bfcode{aboutContext}}{\emph{agentMessage}}{}
Контекст страницы сайта ``About''

\end{fulllineitems}

\index{authoriseContext() (mainServer.messageManager.MessageManager static method)}

\begin{fulllineitems}
\phantomsection\label{messageManager:mainServer.messageManager.MessageManager.authoriseContext}\pysiglinewithargsret{\strong{static }\bfcode{authoriseContext}}{\emph{agentMessage}}{}
Контекст пакета авторизации

\end{fulllineitems}

\index{createContext() (mainServer.messageManager.MessageManager static method)}

\begin{fulllineitems}
\phantomsection\label{messageManager:mainServer.messageManager.MessageManager.createContext}\pysiglinewithargsret{\strong{static }\bfcode{createContext}}{\emph{agentMessage}}{}
Метод для создания контекста для генерации ответных пакетов/страниц

\end{fulllineitems}

\index{createEventContext() (mainServer.messageManager.MessageManager static method)}

\begin{fulllineitems}
\phantomsection\label{messageManager:mainServer.messageManager.MessageManager.createEventContext}\pysiglinewithargsret{\strong{static }\bfcode{createEventContext}}{\emph{agentMessage}}{}
Контекст пакета создания событий

\end{fulllineitems}

\index{delMetooContext() (mainServer.messageManager.MessageManager static method)}

\begin{fulllineitems}
\phantomsection\label{messageManager:mainServer.messageManager.MessageManager.delMetooContext}\pysiglinewithargsret{\strong{static }\bfcode{delMetooContext}}{\emph{agentMessage}}{}
Контекст пакета запроса отказа на событие

\end{fulllineitems}

\index{deleteEventContext() (mainServer.messageManager.MessageManager static method)}

\begin{fulllineitems}
\phantomsection\label{messageManager:mainServer.messageManager.MessageManager.deleteEventContext}\pysiglinewithargsret{\strong{static }\bfcode{deleteEventContext}}{\emph{agentMessage}}{}
Контекст пакета удаления событий

\end{fulllineitems}

\index{devContext() (mainServer.messageManager.MessageManager static method)}

\begin{fulllineitems}
\phantomsection\label{messageManager:mainServer.messageManager.MessageManager.devContext}\pysiglinewithargsret{\strong{static }\bfcode{devContext}}{\emph{agentMessage}}{}
Контекст cтраницы сайта для разработки и тестирования

\end{fulllineitems}

\index{downloadContext() (mainServer.messageManager.MessageManager static method)}

\begin{fulllineitems}
\phantomsection\label{messageManager:mainServer.messageManager.MessageManager.downloadContext}\pysiglinewithargsret{\strong{static }\bfcode{downloadContext}}{\emph{agentMessage}}{}
Контекст cтраницы сайта ``Download''

\end{fulllineitems}

\index{eventsContext() (mainServer.messageManager.MessageManager static method)}

\begin{fulllineitems}
\phantomsection\label{messageManager:mainServer.messageManager.MessageManager.eventsContext}\pysiglinewithargsret{\strong{static }\bfcode{eventsContext}}{\emph{agentMessage}}{}
Контекст пакета запроса событий

\end{fulllineitems}

\index{mainContext() (mainServer.messageManager.MessageManager static method)}

\begin{fulllineitems}
\phantomsection\label{messageManager:mainServer.messageManager.MessageManager.mainContext}\pysiglinewithargsret{\strong{static }\bfcode{mainContext}}{\emph{agentMessage}}{}
Контекст основной страницы сайта

\end{fulllineitems}

\index{metooContext() (mainServer.messageManager.MessageManager static method)}

\begin{fulllineitems}
\phantomsection\label{messageManager:mainServer.messageManager.MessageManager.metooContext}\pysiglinewithargsret{\strong{static }\bfcode{metooContext}}{\emph{agentMessage}}{}
Контекст пакета запроса подписывания на событие

\end{fulllineitems}

\index{modMetooContext() (mainServer.messageManager.MessageManager static method)}

\begin{fulllineitems}
\phantomsection\label{messageManager:mainServer.messageManager.MessageManager.modMetooContext}\pysiglinewithargsret{\strong{static }\bfcode{modMetooContext}}{\emph{agentMessage}}{}
Контекст пакета запроса изменения типа похода на событие

\end{fulllineitems}

\index{modifyEventContext() (mainServer.messageManager.MessageManager static method)}

\begin{fulllineitems}
\phantomsection\label{messageManager:mainServer.messageManager.MessageManager.modifyEventContext}\pysiglinewithargsret{\strong{static }\bfcode{modifyEventContext}}{\emph{agentMessage}}{}
Контекст пакета изменения событий

\end{fulllineitems}

\index{modifyProfileContext() (mainServer.messageManager.MessageManager static method)}

\begin{fulllineitems}
\phantomsection\label{messageManager:mainServer.messageManager.MessageManager.modifyProfileContext}\pysiglinewithargsret{\strong{static }\bfcode{modifyProfileContext}}{\emph{agentMessage}}{}
Контекст пакета редактирования профиля

\end{fulllineitems}

\index{pingContext() (mainServer.messageManager.MessageManager static method)}

\begin{fulllineitems}
\phantomsection\label{messageManager:mainServer.messageManager.MessageManager.pingContext}\pysiglinewithargsret{\strong{static }\bfcode{pingContext}}{\emph{agentMessage}}{}
Контекст пингующего пакета

\end{fulllineitems}

\index{registrateContext() (mainServer.messageManager.MessageManager static method)}

\begin{fulllineitems}
\phantomsection\label{messageManager:mainServer.messageManager.MessageManager.registrateContext}\pysiglinewithargsret{\strong{static }\bfcode{registrateContext}}{\emph{agentMessage}}{}
Контекст пакета регистрации

\end{fulllineitems}

\index{testContext() (mainServer.messageManager.MessageManager static method)}

\begin{fulllineitems}
\phantomsection\label{messageManager:mainServer.messageManager.MessageManager.testContext}\pysiglinewithargsret{\strong{static }\bfcode{testContext}}{\emph{agentMessage}}{}
Контекст пакета тестирования

\end{fulllineitems}

\index{usersContext() (mainServer.messageManager.MessageManager static method)}

\begin{fulllineitems}
\phantomsection\label{messageManager:mainServer.messageManager.MessageManager.usersContext}\pysiglinewithargsret{\strong{static }\bfcode{usersContext}}{\emph{agentMessage}}{}
Контекст пакета запроса пользователей, подписавшихся на событие

\end{fulllineitems}

\index{whatContext() (mainServer.messageManager.MessageManager static method)}

\begin{fulllineitems}
\phantomsection\label{messageManager:mainServer.messageManager.MessageManager.whatContext}\pysiglinewithargsret{\strong{static }\bfcode{whatContext}}{\emph{agentMessage}}{}
Контекст cтраницы сайта ``What is it?''

\end{fulllineitems}


\end{fulllineitems}



\chapter{CHAPTER 5 Session manager}
\label{sessionManager::doc}\label{sessionManager:chapter-5-session-manager}

\section{Система управления сессиями}
\label{sessionManager:id1}\phantomsection\label{sessionManager:module-mainServer.sessionManager}\index{mainServer.sessionManager (module)}
Модуль работы с сессиями
\index{CheckSessionWorker (class in mainServer.sessionManager)}

\begin{fulllineitems}
\phantomsection\label{sessionManager:mainServer.sessionManager.CheckSessionWorker}\pysigline{\strong{class }\code{mainServer.sessionManager.}\bfcode{CheckSessionWorker}}
Класс, переодически подчиющий таблицу сессий
\index{work() (mainServer.sessionManager.CheckSessionWorker method)}

\begin{fulllineitems}
\phantomsection\label{sessionManager:mainServer.sessionManager.CheckSessionWorker.work}\pysiglinewithargsret{\bfcode{work}}{}{}
Вызываемый по таймеру метод

\end{fulllineitems}


\end{fulllineitems}

\index{SessionManager (class in mainServer.sessionManager)}

\begin{fulllineitems}
\phantomsection\label{sessionManager:mainServer.sessionManager.SessionManager}\pysigline{\strong{class }\code{mainServer.sessionManager.}\bfcode{SessionManager}}
Класс для работы с сессиями.Позволяет:
\begin{itemize}
\item {} 
Получить сессию для указанного пользователя

\item {} 
Создать сессию для пользователя

\item {} 
Проверить, существует ли сессия

\item {} 
Получить userId по сессии

\end{itemize}
\index{checkSession() (mainServer.sessionManager.SessionManager static method)}

\begin{fulllineitems}
\phantomsection\label{sessionManager:mainServer.sessionManager.SessionManager.checkSession}\pysiglinewithargsret{\strong{static }\bfcode{checkSession}}{\emph{sessionId}}{}
Метод для проверки существования сессии

\end{fulllineitems}

\index{createSession() (mainServer.sessionManager.SessionManager static method)}

\begin{fulllineitems}
\phantomsection\label{sessionManager:mainServer.sessionManager.SessionManager.createSession}\pysiglinewithargsret{\strong{static }\bfcode{createSession}}{\emph{user}}{}
Метод для создания сессии для пользователя

\end{fulllineitems}

\index{getSessionID() (mainServer.sessionManager.SessionManager static method)}

\begin{fulllineitems}
\phantomsection\label{sessionManager:mainServer.sessionManager.SessionManager.getSessionID}\pysiglinewithargsret{\strong{static }\bfcode{getSessionID}}{\emph{userId}}{}
Метод для получения id сессии по id пользователя

\end{fulllineitems}

\index{getUser() (mainServer.sessionManager.SessionManager static method)}

\begin{fulllineitems}
\phantomsection\label{sessionManager:mainServer.sessionManager.SessionManager.getUser}\pysiglinewithargsret{\strong{static }\bfcode{getUser}}{\emph{sessionId}}{}
Метод для получения user по sessionId

\end{fulllineitems}

\index{getUserId() (mainServer.sessionManager.SessionManager static method)}

\begin{fulllineitems}
\phantomsection\label{sessionManager:mainServer.sessionManager.SessionManager.getUserId}\pysiglinewithargsret{\strong{static }\bfcode{getUserId}}{\emph{sessionId}}{}
Метод для получения userId по sessionId

\end{fulllineitems}


\end{fulllineitems}



\chapter{CHAPTER 6 User manager}
\label{userManager:chapter-6-user-manager}\label{userManager::doc}

\section{Система управления пользователями}
\label{userManager:id1}\phantomsection\label{userManager:module-mainServer.userManager}\index{mainServer.userManager (module)}
Модуль работы с пользователями
\index{UserManager (class in mainServer.userManager)}

\begin{fulllineitems}
\phantomsection\label{userManager:mainServer.userManager.UserManager}\pysigline{\strong{class }\code{mainServer.userManager.}\bfcode{UserManager}}
Класс предоставляющий методы работы с пользователями
\begin{itemize}
\item {} 
Создание аккаунта

\item {} 
Редактирование аккаунта

\item {} 
Подключение

\item {} 
Отключение

\item {} 
Добавить друзей

\item {} 
Удалить друзей

\item {} 
Выдать список друзей

\end{itemize}
\index{addFriends() (mainServer.userManager.UserManager static method)}

\begin{fulllineitems}
\phantomsection\label{userManager:mainServer.userManager.UserManager.addFriends}\pysiglinewithargsret{\strong{static }\bfcode{addFriends}}{\emph{userid}, \emph{session\_id}, \emph{list\_newfriends}}{}
Метод добавления друзей к пользователю

\end{fulllineitems}

\index{connectUser() (mainServer.userManager.UserManager static method)}

\begin{fulllineitems}
\phantomsection\label{userManager:mainServer.userManager.UserManager.connectUser}\pysiglinewithargsret{\strong{static }\bfcode{connectUser}}{\emph{login}, \emph{password}}{}
Метод для подключения пользователя к серверу

\end{fulllineitems}

\index{createAccount() (mainServer.userManager.UserManager static method)}

\begin{fulllineitems}
\phantomsection\label{userManager:mainServer.userManager.UserManager.createAccount}\pysiglinewithargsret{\strong{static }\bfcode{createAccount}}{\emph{new\_login}, \emph{new\_password}}{}
Метод для создания аккаунта

\end{fulllineitems}

\index{deleteFriends() (mainServer.userManager.UserManager static method)}

\begin{fulllineitems}
\phantomsection\label{userManager:mainServer.userManager.UserManager.deleteFriends}\pysiglinewithargsret{\strong{static }\bfcode{deleteFriends}}{\emph{userid}, \emph{session\_id}, \emph{list\_exfriends}}{}
Метод удаления друзей у пользователя

\end{fulllineitems}

\index{disconnectUser() (mainServer.userManager.UserManager static method)}

\begin{fulllineitems}
\phantomsection\label{userManager:mainServer.userManager.UserManager.disconnectUser}\pysiglinewithargsret{\strong{static }\bfcode{disconnectUser}}{\emph{userid}, \emph{session\_id}}{}
Метод для отключения пользователя от сервера

\end{fulllineitems}

\index{editAccount() (mainServer.userManager.UserManager static method)}

\begin{fulllineitems}
\phantomsection\label{userManager:mainServer.userManager.UserManager.editAccount}\pysiglinewithargsret{\strong{static }\bfcode{editAccount}}{\emph{userid}, \emph{session\_id}, \emph{list\_changes}}{}
Метод для редактирования аккаунта

\end{fulllineitems}

\index{getListFriends() (mainServer.userManager.UserManager static method)}

\begin{fulllineitems}
\phantomsection\label{userManager:mainServer.userManager.UserManager.getListFriends}\pysiglinewithargsret{\strong{static }\bfcode{getListFriends}}{\emph{userid}, \emph{session\_id}}{}
Метод получения списка друзей

\end{fulllineitems}


\end{fulllineitems}



\chapter{CHAPTER 7 Event manager}
\label{eventManager::doc}\label{eventManager:chapter-7-event-manager}

\section{Система управления событиями}
\label{eventManager:id1}\phantomsection\label{eventManager:module-mainServer.eventManager}\index{mainServer.eventManager (module)}\index{EventManager (class in mainServer.eventManager)}

\begin{fulllineitems}
\phantomsection\label{eventManager:mainServer.eventManager.EventManager}\pysigline{\strong{class }\code{mainServer.eventManager.}\bfcode{EventManager}}
Класс, предоставлящий возможности:
\begin{itemize}
\item {} 
Создания событий;

\item {} 
Запроса событий по критерию;

\item {} 
Изменения событий;

\item {} 
Удаления событий;

\end{itemize}
\index{checkEvent() (mainServer.eventManager.EventManager static method)}

\begin{fulllineitems}
\phantomsection\label{eventManager:mainServer.eventManager.EventManager.checkEvent}\pysiglinewithargsret{\strong{static }\bfcode{checkEvent}}{\emph{eventId}}{}
Метод для проверки существованияя события(по id события)

\end{fulllineitems}

\index{createEvent() (mainServer.eventManager.EventManager static method)}

\begin{fulllineitems}
\phantomsection\label{eventManager:mainServer.eventManager.EventManager.createEvent}\pysiglinewithargsret{\strong{static }\bfcode{createEvent}}{\emph{sessionId}, \emph{eventArgs}}{}
Метод для создания событий (по сессии и списку аргументов)

\end{fulllineitems}

\index{deleteEvent() (mainServer.eventManager.EventManager static method)}

\begin{fulllineitems}
\phantomsection\label{eventManager:mainServer.eventManager.EventManager.deleteEvent}\pysiglinewithargsret{\strong{static }\bfcode{deleteEvent}}{\emph{sessionId}, \emph{eventId}}{}
Метод для удаления событий (по сессии и Id сессии)

\end{fulllineitems}

\index{getEvents() (mainServer.eventManager.EventManager static method)}

\begin{fulllineitems}
\phantomsection\label{eventManager:mainServer.eventManager.EventManager.getEvents}\pysiglinewithargsret{\strong{static }\bfcode{getEvents}}{\emph{sessionId}, \emph{conditionals}}{}
Метод для запроса событий (по сессии и запросу)

\end{fulllineitems}

\index{modifyEvent() (mainServer.eventManager.EventManager static method)}

\begin{fulllineitems}
\phantomsection\label{eventManager:mainServer.eventManager.EventManager.modifyEvent}\pysiglinewithargsret{\strong{static }\bfcode{modifyEvent}}{\emph{sessionId}, \emph{eventId}, \emph{eventArgs}}{}
Метод для редактирования событий (по сессии и списку аргументов)

\end{fulllineitems}


\end{fulllineitems}



\chapter{CHAPTER 8 MeToo manager}
\label{MeTooManager:chapter-8-metoo-manager}\label{MeTooManager::doc}

\section{Система управления походами на событие}
\label{MeTooManager:id1}\phantomsection\label{MeTooManager:module-mainServer.MeTooManager}\index{mainServer.MeTooManager (module)}
Модуль, отвечающий за работу с походами на события.
\index{MeTooManager (class in mainServer.MeTooManager)}

\begin{fulllineitems}
\phantomsection\label{MeTooManager:mainServer.MeTooManager.MeTooManager}\pysigline{\strong{class }\code{mainServer.MeTooManager.}\bfcode{MeTooManager}}
Класс, позволяющий узнать, кто идет на конкретное событие,

пойти самому, изменить тип похода или отказаться от события
\index{delMeToo() (mainServer.MeTooManager.MeTooManager method)}

\begin{fulllineitems}
\phantomsection\label{MeTooManager:mainServer.MeTooManager.MeTooManager.delMeToo}\pysiglinewithargsret{\bfcode{delMeToo}}{\emph{sessionId}, \emph{eventId}}{}
Метод отказа от события

\end{fulllineitems}

\index{getUsersbyEvent() (mainServer.MeTooManager.MeTooManager method)}

\begin{fulllineitems}
\phantomsection\label{MeTooManager:mainServer.MeTooManager.MeTooManager.getUsersbyEvent}\pysiglinewithargsret{\bfcode{getUsersbyEvent}}{\emph{sessionId}, \emph{eventId}}{}
Метод, позволяющий узнать, кто идет на конкретное событие

\end{fulllineitems}

\index{meToo() (mainServer.MeTooManager.MeTooManager method)}

\begin{fulllineitems}
\phantomsection\label{MeTooManager:mainServer.MeTooManager.MeTooManager.meToo}\pysiglinewithargsret{\bfcode{meToo}}{\emph{sessionId}, \emph{eventId}, \emph{metooTypeId}}{}
Метод, для того, чтобы пойти на событие

\end{fulllineitems}

\index{modMeToo() (mainServer.MeTooManager.MeTooManager method)}

\begin{fulllineitems}
\phantomsection\label{MeTooManager:mainServer.MeTooManager.MeTooManager.modMeToo}\pysiglinewithargsret{\bfcode{modMeToo}}{\emph{sessionId}, \emph{eventId}, \emph{metooTypeId}}{}
Метод для измения типа похода

\end{fulllineitems}


\end{fulllineitems}



\chapter{CHAPTER 9 Служебные модули}
\label{config:chapter-9}\label{config::doc}

\section{Конфигурационные настройки сервера}
\label{config:module-mainServer.configurationManager}\label{config:id1}\index{mainServer.configurationManager (module)}\index{ConfigurationManager (class in mainServer.configurationManager)}

\begin{fulllineitems}
\phantomsection\label{config:mainServer.configurationManager.ConfigurationManager}\pysigline{\strong{class }\code{mainServer.configurationManager.}\bfcode{ConfigurationManager}}
Класс конфигурационных настроек сервера
\index{loopDeleteSessionInterval() (mainServer.configurationManager.ConfigurationManager static method)}

\begin{fulllineitems}
\phantomsection\label{config:mainServer.configurationManager.ConfigurationManager.loopDeleteSessionInterval}\pysiglinewithargsret{\strong{static }\bfcode{loopDeleteSessionInterval}}{}{}
Метод, возвращающий таймаут цикла удаления старых сессий

\end{fulllineitems}

\index{serverVersion() (mainServer.configurationManager.ConfigurationManager static method)}

\begin{fulllineitems}
\phantomsection\label{config:mainServer.configurationManager.ConfigurationManager.serverVersion}\pysiglinewithargsret{\strong{static }\bfcode{serverVersion}}{}{}
Метод, возвращающий версию сервера

\end{fulllineitems}

\index{sessionDeleteInterval() (mainServer.configurationManager.ConfigurationManager static method)}

\begin{fulllineitems}
\phantomsection\label{config:mainServer.configurationManager.ConfigurationManager.sessionDeleteInterval}\pysiglinewithargsret{\strong{static }\bfcode{sessionDeleteInterval}}{}{}
Метод, возвращающий таймаут удаления старых сессий

\end{fulllineitems}


\end{fulllineitems}



\section{Подсистема управления временем сервера}
\label{config:id2}\label{config:module-mainServer.timeManager}\index{mainServer.timeManager (module)}\index{TimeManager (class in mainServer.timeManager)}

\begin{fulllineitems}
\phantomsection\label{config:mainServer.timeManager.TimeManager}\pysigline{\strong{class }\code{mainServer.timeManager.}\bfcode{TimeManager}}
Класс, предоставляющий методы для работы со временем
\index{getTime() (mainServer.timeManager.TimeManager static method)}

\begin{fulllineitems}
\phantomsection\label{config:mainServer.timeManager.TimeManager.getTime}\pysiglinewithargsret{\strong{static }\bfcode{getTime}}{}{}
Метод для получения текущего времени

\end{fulllineitems}


\end{fulllineitems}

\index{Worker (class in mainServer.timeManager)}

\begin{fulllineitems}
\phantomsection\label{config:mainServer.timeManager.Worker}\pysigline{\strong{class }\code{mainServer.timeManager.}\bfcode{Worker}}
Базовый класс для выполнения переодических действий
\index{doWork() (mainServer.timeManager.Worker method)}

\begin{fulllineitems}
\phantomsection\label{config:mainServer.timeManager.Worker.doWork}\pysiglinewithargsret{\bfcode{doWork}}{\emph{delay}}{}
Метод, вызываемый переодически и запускающий метод выполняемого действия

\end{fulllineitems}

\index{work() (mainServer.timeManager.Worker method)}

\begin{fulllineitems}
\phantomsection\label{config:mainServer.timeManager.Worker.work}\pysiglinewithargsret{\bfcode{work}}{}{}
Абстрактный метод переодически выполняемого действия

\end{fulllineitems}


\end{fulllineitems}



\section{Подсистема управления ошибками}
\label{config:id3}\label{config:module-mainServer.errorManager}\index{mainServer.errorManager (module)}
Модуль, отвечающий за работу с ошибками.
\index{ErrorManager (class in mainServer.errorManager)}

\begin{fulllineitems}
\phantomsection\label{config:mainServer.errorManager.ErrorManager}\pysigline{\strong{class }\code{mainServer.errorManager.}\bfcode{ErrorManager}}
Класс работы с ошибками
\index{getErrorCode() (mainServer.errorManager.ErrorManager static method)}

\begin{fulllineitems}
\phantomsection\label{config:mainServer.errorManager.ErrorManager.getErrorCode}\pysiglinewithargsret{\strong{static }\bfcode{getErrorCode}}{}{}
Метод получения кода ошибки

\end{fulllineitems}


\end{fulllineitems}



\chapter{Indices and tables}
\label{index:indices-and-tables}\begin{itemize}
\item {} 
\emph{genindex}

\item {} 
\emph{modindex}

\item {} 
\emph{search}

\end{itemize}


\renewcommand{\indexname}{Python Module Index}
\begin{theindex}
\def\bigletter#1{{\Large\sffamily#1}\nopagebreak\vspace{1mm}}
\bigletter{m}
\item {\texttt{mainServer.configurationManager}}, \pageref{config:module-mainServer.configurationManager}
\item {\texttt{mainServer.errorManager}}, \pageref{config:module-mainServer.errorManager}
\item {\texttt{mainServer.eventManager}}, \pageref{eventManager:module-mainServer.eventManager}
\item {\texttt{mainServer.messageManager}}, \pageref{messageManager:module-mainServer.messageManager}
\item {\texttt{mainServer.MeTooManager}}, \pageref{MeTooManager:module-mainServer.MeTooManager}
\item {\texttt{mainServer.models}}, \pageref{models:module-mainServer.models}
\item {\texttt{mainServer.sessionManager}}, \pageref{sessionManager:module-mainServer.sessionManager}
\item {\texttt{mainServer.timeManager}}, \pageref{config:module-mainServer.timeManager}
\item {\texttt{mainServer.userManager}}, \pageref{userManager:module-mainServer.userManager}
\item {\texttt{mainServer.views}}, \pageref{view:module-mainServer.views}
\end{theindex}

\renewcommand{\indexname}{Index}
\printindex
\end{document}
